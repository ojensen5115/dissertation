The protocol in use today by contactless credit cards is insecure.
Making use of the NFC channel is a natural choice for a contactless payment protocol:
    NFC is a wireless channel, has a very short range, permits for arbitrary computation, while needing no power source on the card.
However, over-reliance on NFC's short range has led to unfortunate assumptions in the contactless credit card protocol.
For example, a contactles credit card assumes (sometimes incorrectly) that its ability to receive a solicitation implies that its cardholder has made a conscious effort to enable communication.

In this dissertation, we begin by examining the contactless credit card protocol currently in use.
We conclude that transactions are vulnerable to four broad categories of ``outsider'' attacks:
  eavesdropping, skimming attacks, relay attacks, attacks facilitated by compromised points of sale.
Each of these attacks can be prevented through the use of a challenge-response protocol,
    which typically relies on some form of cryptographic hash function.
However, contactless credit cards operate within a strong set of constraints:
    in order to combat credit card fraud, cards need to be disposably cheap to manufacture.
Thus, we consider a hash function to be prohibitively expensive.
Instead, we use stepwise refinement to construct a challenge response protocol which minimizes computation that occurs on the card.

Next, we consider malicious retailers.
Malicious retailers are in control of the device with which credit cards interface,
    and we identify and discuss two attacks which may be carried out.
Both of these attacks are predicated on the customer's lack of involvement in the protocol, beyond simply allowing it to occur.
The emergence of Electronic Wallet applications (such as Android Pay and Apple Wallet) provide an interface between customer and card,
    which we make use of to involve the customer in the protocol, defend against malicious retailers.
This protocol is backwards compatible with physical credit cards, which do not gain protection from malicious retailers,
    but may still participate in the protocol and enjoy protection from malicious outsiders.

Finally, we determine three desirable properties of a credit card protocol beyond preventing fraud:
    \emph{unlinkability}, preventing retailers from linking purchases made by the same card;
    \emph{use of existing infrastructure}, without requiring retailers to replace or modify their points of sale;
    \emph{use of any credit card}, without requiring any special cards or participation from the issuing bank.
We design the Unlinkable Protocol for Electronic Wallets, which provides protection from malicious outsiders as well as malicious retailers,
    while upholding unlinkability, using existing point of sale infrastructure, and allowing for the use of any credit card.
