The protocol in use today for contactless (NFC) credit card transactions is insecure.

The use of NFC is an excellent choice for a contactless payment protocol:
    NFC is a wireless channel, has a very short range, permits for arbitrary computation while needing no power source on the card.
However, its short range has led to unfortunate assumptions on the part of the protocol designers.
For example, the protocol assumes that proximity is tantamount to an intent to pay for something:
    a credit card assumes that its ability to receive a solicitation implies that its cardholder has made a conscious effort to enable communication.
Furthermore, a contactless credit card cannot be ``switched off'', which results in a need for constant vigilance on the part of the cardholder:
    any party in control of an NFC-capable device (e.g. a smart phone) may interface with the credit card at any time,
    provided that they are briefly in range.

We examine this protocol in detail.
Protected by nothing more than the short range of NFC and the inclusion of a single-use token, credit card information is transmitted in cleartext.
We determine that contactless credit card transactions are vulnerable to four broad categories of ``outsider'' attacks:
    \emph{eavesdropping}, wherein a malicious actor places a listening device within range of legitimate transmissions;
    \emph{skimming attacks}, wherein a malicious actor initiates a transaction with a credit card without its cardholder's knowledge or consent;
    \emph{relay attacks}, wherein two malicious actors relay information out-of-band, bypassingn the proximity assurance of NFC;
    \emph{attacks facilitated by compromised points of sale}, where a point of sale has been compromised and transmits sensitive information to a third party.
We discuss these four classes of attacks in some detail, then construct a replacement protocol which defends against them.

Each of these attacks can be prevented through the use of a challenge-response protocol, and such protocols are not new.
However, contactless credit cards operate within a strong set of constraints:
    in order to combat credit card fraud, cards need to be disposably cheap to manufacture.
Most challenge-response protocols rely on some form of cryptographic hash function.
By contrast, we use a process called stepwise refinement to construct a challenge response protocol which minimizes computation which occurs on the card,
    considering even hash functions to be prohibitively expensive.

We then consider the problem posed by malicious retailers.
Being in control of the device with which credit cards interface, malicious retailers are in a unique position to carry out several attacks.
At the most basic level, a retailer may increase the price of a transaction without informing the customer.
More advanced, a retailer and an accomplice may form a communication bridge between a victim credit card and a victim point of sale.
This second attack is particularly insidioius as the malicious parties leave no traces with either of the victims.

Both of these attacks are predicated on the fact that, beyond simply allowing it to occur, the customer cannot interact with the credit card protocol in any way.
Thus, we augment our previous protocol to allow the customer to bind a card's response to a particular price.
In so doing, the first attack is prevented, and the motivation for performing the second attack is removed.
In order to allow a customer to participate in this protocol, he must be able to interface with the credit card in some way.
We recognize the entrance of smart phone applications such as Android Pay and Apple Wallet, which could enable precisely such interaction.
However, the protocol remains backwards compatible, in that while the user of a physical credit card will not gain protection against malicious retailers,
    he may still participate in the protocol and enjoy protection from malicious outsiders.

In addition, we determine three desirable properties of a contactless credit card protocol beyond simply preventing fraud:
    \emph{unlinkability}, preventing retailers from correlating purchases as having made by the same customer;
    \emph{use of existing infrastructure}, without requiring retailers to replace or modify their points of sale;
    \emph{use of any credit card}, without requiring any special cards or participation from the issuing bank.
Finally we design the Unlinkable CC Protocol for electronic wallets, which provides
    protection from malicious outsiders,
    protection from malicious retailers,
    and upholds unlinkability between purchases,
    all while using existing point of sale infrastructure and allowing for the use of any credit card.
