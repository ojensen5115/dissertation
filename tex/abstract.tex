The protocol in use today by contactless credit cards is insecure.
The credit card industry has chosen to use the NFC channel for contactless credit card transactions.
However, reliance on NFC's short range has led to unfortunate assumptions in the contactless credit card protocol.
For example, a contactles credit card assumes (sometimes incorrectly) that its ability to receive a solicitation implies the cardholder's intent to purchase.
In this dissertation, we begin by examining the protocol currently in use,
    and present a family of replacement protocols to defend against its deficiencies.

We first consider ``outsider'' attacks, and determine that transactions are vulnerable to
    eavesdropping, skimming attacks, relay attacks, attacks facilitated by compromised points of sale.
While standard cryptographic approaches could defend against these attacks,
    increasing computational power on credit cards leads to higher manufacturing costs.
Furthermore, altering the protocol structure requires retailer cooperation,
    and so preserving \emph{linkability} (retailers' ability to identify purchases made with the same card) is an explicit goal.
We use stepwise refinement to construct the Externally Secure CC Protocol using single-use ``charge tokens'' verifiable by the bank and recognizable by the retailer,
    while minimizing computation that occurs on the card.

Next, we consider malicious retailers.
Retailers are in control of the device with which credit cards communicate,
    and we identify and discuss two attacks which may be carried out by a malicious retailer.
Both of these attacks rely on charging the customer a different price than is displayed,
    and are predicated on the customer's lack of involvement in the protocol beyond simply allowing it to occur.
The emergence of Electronic Wallet applications (such as Android Pay and Apple Wallet) provide a channel between customer and card.
We use this channel to construct the Secure CC Protocol,
    allowing the customer to confirm the purchase price and binding it to the charge token.

Finally, we determine three desirable properties of a credit card protocol beyond preventing fraud:
    \emph{unlinkability}, preventing retailers from identifying purchases made by the same card;
    \emph{use of existing infrastructure}, without requiring retailers to replace or modify their points of sale;
    \emph{use of any credit card}, without requiring any special cards or participation from the issuing bank.
In response, we design the Unlinkable Electronic Wallet Protocol,
    leveraging techniques from the Secure CC Protocol to guard against malicious outsiders and retailers,
    while tunneling unlinkable charge tokens through the protocol in use today.