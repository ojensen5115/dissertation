\section{Concluding Remarks}

The family of protocols presented in this dissertation are a strong step forwards from the status quo.
Protecting the contactless credit card protocol from malicious outsiders should be considered an urgent priority.
Protecting the protocol from malicious retailers is less urgent, but is nonetheless important given the severity of the attacks which may be carried out.
Whether to adopt unlinkability should be a customer choice:
    we do not presume to dictate the best move for the industry in this regard, but should a customer be privacy conscious, we seek to place this decision in the customer's hands.
That being said, the protocols do carry limitations.

The Externally Secure CC Protocol and the Secure CC Protocol both require large changes to existing point of sale infrastructure.
This poses a significant barrier to adoption, as adopting either of these protocols requires replacing all point of sale devices across all retailers.
They also represent a significant change in a generally change-averse industry.

The Unlinkable Wallet Protocol operates using existing point of sale infrastructure.
However, the Wallet Server must maintain an index of the next expected token value for each card into its credit card database.
This requires rebuilding indexes over the database on every transaction, which may result in a performance bottleneck.
The Unlinkable Wallet Protocol may also yield some customer confusion:
    bank statements are no longer as useful, as all transactions will appear to be made out to the Wallet Server service.
Furthermore, this protocol hampers any systems which require customers to identify themselves via previously used credit card,
    commonly seen with retailreturns and ticket collection kiosks.
The Unlinkable Wallet Protocol renders this form of identification impossible by design.
