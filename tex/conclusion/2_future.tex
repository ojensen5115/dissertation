\section{Future Work}

The work described in this dissertation suggests the following avenues that merit further research:

\begin{enumerate}
    \item Protocols relying on an iCVV (i.e. the Insecure CC Protocol currently in use, the Externally Secure CC Protocol, and the Secure CC Protocol) may suffer de-synchronization.
        This happens when Card Information messages are generated and sent by the card, but resulting Charge Request messages not reach the bank
            (as a result of persistent network failures at the point of sale or as a result of repeated skimming attempts).
        When the card's internal iCVV has progressed past the number of failures anticipated by the bank, contactless functionality of the card becomes inoperative.
        The construction of a re-synchronization protocol for iCVV states would alleviate this issue.

    \item The Unlinkable Wallet Protocol involves fast lookup on a table of registered credit card records by the Wallet Server, via the corresponding single-use token.
        This table contains every credit card registered with the Wallet Server, and may therefore contain millions of rows.
        Our description of the protocol suggests pre-computing the next expected token for each card, and using this value as an index over the table.
        This value is replaced on each transaction, and as a consequence results in frequent index rebuilding, which may result in a performance bottleneck at scale.
        The construction of a better data-structure for use on the Wallet Server to represent registered credit cards,
            allowing fast record lookup given a frequently changing lookup key, would provide a significant performance gain and allow the system to scale more effectively.

    \item While unlinkable transactions are a significant gain to customer privacy, they present drawbacks in specific scenarios.
        For example, unlinkable transactions hamper any system in which customers identify themselves via a previously used credit card.
        This is commonly seen in retail returns and ticket collection kiosks, where the customer presents the credit card used to make the purchase.
        The Unlinkable Wallet Protocol renders this form of identification impossible by design.
        The construction of a protocol in which the Wallet Application can allow the customer to choose, on a per-purchase basis,
            whether the transaction should be linkable or unlinkable provides the flexibility needed to work around this issue.
\end{enumerate}

The family of protocols presented in this dissertation represents a strong step above the status quo.
We consider protection from malicious outsiders to be an urgent priority.
Protecting customers from malicious retailers is less urgent, but is nonetheless important given the severity of attacks which malicious retailers may perpetrate.
Whether to adopt a protocol providing linkability or unlinkability should be a customer choice:
    while we do not presume to dictate the best move for the industry in this regard,
    we place the decision of a customer's privacy in the customer's hands rather than in those of the retailers.
