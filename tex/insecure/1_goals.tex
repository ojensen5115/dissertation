\section{Goals}
\label{sec:insecure-goals}

A credit card payment system has five fundamental principals:

\begin{enumerate}
\item A \textbf{customer} who wants to make a purchase.
\item A \textbf{bank} at which the customer has an account.
\item A \textbf{credit card} issued by the bank to the customer.
\item A \textbf{retailer} from whom the customer wishes to make the purchase.
\item A \textbf{point of sale} controlled and initialized by the retailer.
	It displays the purchase price to the customer, and communicates with both the credit card and its issuing bank to coordinate the transaction.
\end{enumerate}

The underlying goal of any credit card payment protocol is to enable the customer and the retailer to negotiate a transaction,
    after which the customer's bank debits the appropriate funds from the customer's account and issues a payment to the retailer.

Traditional magnetic-stripe credit card systems have been in operation in this space for many years, but they face several important drawbacks:
    it is easy to accidentally de-magnetize your credit card, and
    dirty or corroded contacts on the point of sale can make even a well-magnetized card difficult to read.
As a result, it is not at all uncommon for a retailer to need to swipe a credit card multiple times before a successful read occurs.

The primary goal of the Insecure CC Protocol is to solve these drawbacks by using a contactless (i.e. wireless) communication channel.
However, due to the reduction of control that a contactless solution presents, this protocol has a secondary goal as well:
    to prevent a malicious actor from cloning a credit card simply by querying its contents.
