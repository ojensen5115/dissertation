\section{Goals}
\label{sec:insecure-goals}

A credit card payment system has five fundamental principals:

\begin{enumerate}
\item A \textbf{Customer} who wants to make a purchase.
\item A \textbf{Bank} at which the Customer has an account.
\item A \textbf{Credit Card} issued by the Bank to the Customer.
\item A \textbf{Retailer} from whom the Customer wishes to make the purchase.
\item A \textbf{Point of Sale} controlled and initialized by the Retailer.
	It displays the purchase price to the Customer, and communicates with both the Credit Card and its issuing Bank to coordinate the transaction.
\end{enumerate}

The underlying goal of any credit card payment protocol is to enable the Customer and the Retailer to negotiate a transaction,
    after which the Customer's Bank debits the appropriate funds from the Customer's account and issues a payment to the Retailer.

Traditional magnetic-stripe credit card systems have been in operation in this space for many years, but they face several important drawbacks:
    it is easy to accidentally de-magnetize your credit card, and
    dirty or corroded contacts on the Point of Sale can make even a well-magnetized card difficult to read.
As a result, it is not at all uncommon for a Retailer to need to swipe a credit card multiple times before a successful read occurs.

The primary goal of the Insecure CC Protocol is to solve these drawbacks by using a contactless (i.e. wireless) communication channel.
However, due to the reduction of control that a contactless solution presents, this protocol has a secondary goal as well:
    to prevent a malicious actor from cloning a Credit Card simply by querying its contents.
