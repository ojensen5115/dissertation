\subsection{Collision Analysis of the Protocol}
\label{sec:collisions_and_simulations}

Every transaction processed by the Wallet Server service updates a card's Account Hash to a new 80-bit value.
If at any time two cards possess the same Account Hash, the Wallet Server can no longer process purchases for either card,
    as a Charge Request message cannot uniquely identify a card.
While hash functions such as those in the SHA-2 family are collision resistant, this does not mean that collisions cannot occur.
Furthermore, truncating the Account Hash down to 80 bits greatly increases the probability of collisions.

We wrote a simulator to model Wallet Server behavior, in order to ensure that messages contained sufficient information for the protocol to succeed,
    and to monitor for collisions.
We seeded it with 10 million credit cards, and processed several million transactions.
During the simulation, no collisions were observed.

Modeling this behavior mathematically, we note that the Birthday Paradox plays a role in determining the number of credit cards that a database can contain before expecting a hash collision:
the expected number cards registered before observing a collision is approximately
$\sqrt{\frac{\pi}{2} \cdot 2^{80}} \approx 1.4 \cdot 10^{12}$.
However, registering a credit card is by definition an event in which the Wallet Application is connected to the Internet (and thus to the Wallet Server).
As such, collisions during registration are not a concern, as a differing seed or counter value can simply be chosen and synchronized between the Wallet and the Server.
As a result, provided that there are fewer than $2^{79}$ (approximately $6.04 * 10^{24}$) credit cards registered, card registration is computationally easy (expecting no more than two attempts) without causing a collision.

A credit card transaction is then modeled by selecting a single hash and replacing it.
In such a transaction, the Birthday Paradox is not relevant:
intuitively, updating Hash number 83 cannot cause Hashes 12 and 41 to collide.
As a result, given a database of $n$ Account Hashes, the expected number of transactions before a collision occurs is
$\frac{2^{80}}{n}$.
Using the 10 million cards from our simulation, this implies that the expected number of transactions before a collision occurs is
$\frac{2^{80}}{10,000,000} \approx 1.2 \cdot 10 ^ {17}$, or approximately 120 million billion transactions.
It is thus hardly surprising that no collisions were observed in simulation.

As such, the inconvenience of rendering both cards inoperative until at least one of them reconnects to the Internet is not particularly onerous.