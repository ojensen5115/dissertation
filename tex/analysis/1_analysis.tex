\section{Analysis of the Unlinkable Wallet Protocol}
\label{sec:analysis}

Simply by using an Electronic Wallet, the customer is protected from skimmers and relay attackers ``for free'',
    because the phone must be unlocked and ready to transmit in order to respond to a Solicitation message.
The Unlinkable Wallet Protocol provides protection against eavesdroppers and compromised points of sale,
    because each message consists of two single-use tokens which lose all value after the transaction takes place.

Due to the inclusion of the price validation hash (the last 13 bits of the 93-bit message),
    the Unlinkable Wallet Protocol also defends against Malicious Retailers by binding the charge token to a price confirmed by the customer.

Furthermore, the Unlinkable Wallet Protocol provides the customer with the privacy property of \emph{Unlinkability} from the retailer.
In this protocol, retailers lose the ability to use credit card numbers as a method to correlate purchases and track customer purchasing profiles through purchase records.

In our initial publications \cite{sccp} \cite{???}, we considered Unlinkability an explicit non-goal:
  for deployment, the protocol required extensive cooperation from retailers who currently enjoy this property.
The Unlinkable Wallet Protocol requires no such cooperation from retailers on the payment industry,
    which means that bowing to retailers' wishes is not necessary.

Note that the information correlating an individual credit card to an individual purchase is not lost:
    it is now learned by the Wallet Server provider instead.
This is potentially valuable information, and thus is a strong incentive to be in the position of Electronic Wallet provider.

Indeed, Electronic Wallet providers find themselves in a very privileged position:
    by inserting themselves between retailers and banks, they appropriate knowledge from retailers regarding transaction correlation
    and they appropriate knowledge from banks regarding transaction locations\footnote{
This appears to be consistent with the operation of Electronic Wallet applications like Android Pay and Apple Wallet:
    Android Pay has announced partnerships with some banks,
    which results in your credit card's purchases through Android Pay continuing to earn rewards normally offered by your credit card
    (e.g. 5 points per dollar spent on gas, 3 points per dollar spent on groceries, etc.).
    This suggests that, lacking a partnership with Electronic Wallet providers, banks lose information on the nature and location of purchases.}.

Even so, the Unlinkable Wallet Protocol described in Chapter \ref{cha:unlinkable} protects consumer privacy
    by separating knowledge of itemized purchases and correlation between purchases.
It is potentially valuable for an entity to be able to answer the questions \emph{Where do you spend your money? What do you spend it on?}

Currently, your bank has complete information on where you spend your money,
    and a retailer at which you shop has complete information on what you spend your money on (at that retailer).
The Unlinkable Wallet Protocol breaks part of this information away from both parties:
    all purchases appear to the bank as being made out to the Electronic Wallet service, and
    all purchases appear to the retailer as being made by a unique credit card.

In all scenarios, customer's privacy is bolstered, as it is now more difficult and more expensive for any entity to track the customer's purchases.
Banks and retailers can potentially partner with the Electronic Wallet provider in order to regain access to this information,
    but doing so requires effort and cost, thereby raising the bar on who will choose to actually do so.