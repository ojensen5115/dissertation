\section{Related Works}
\label{sec:related}

Madlmayr et al. analyze the state of NFC communication privacy \cite{madlmayr2008nfc},
    focusing not only the security and privacy of communications, but also the continued operability of device and host controller.
They enumerate and discusse the viability and consequences of a number of attacks, but the discussion pertains only to channel security.
Kortvedt further explores the problem of eavesdropping on NFC communications \cite{kortvedt2009securing},
    suggesting various improvements such as a symmetric encryption solution with a strong mutual authentication,
    using ``Over-the-Air Programming'' (OTA) as a solution for key management.
Both works \cite{kortvedt2009securing} and \cite{madlmayr2008nfc} focus on channel security, and thus are effective against channel attacks such as eavesdropping.
However, the primary attacks targeting contactless payment systems today
    (e.g. skimmers, relays, compromised points of sale, and attacks perpetrated by malicious retailers) do not exploit weaknesses of the channel.
As such, neither approach is effective at protecting NFC credit card payments.

Haselsteine and Breitfu{\ss} provide a broad survey in \cite{haselsteiner2006security} of various attacks and defenses applicable to protocols built on the NFC channel.
Similarly to \cite{madlmayr2008nfc} and \cite{kortvedt2009securing}, they focus on securing the channel itself from attackers,
    suggesting that NFC participants perform a key-exchange protocol such as Diffie-Helmann \cite{diffiehellman},
    then use this derived secret key to establish a secure channel.
As a result, this approach also falls short of protecting NFC credit card payments, for the same reason.

In \cite{lee2012nfc}, Lee provides some analysis of relay and skimming attacks on NFC credit card transactions.
The stated goal of this work is to demonstrate the simplicity of performing these attacks,
    emphasizing that they are easily performed by the general public (having little-to-no knowledge of NFC or credit card protocols).
To this end, he presents an Android application \emph{NFCProxy} \cite{NFCProxy} which implements these attacks.
Indeed, the application is easily installed, and transforms contactless credit card skimming into no more than a button-push endeavor.

Drimer and Murdoch \cite{Drimer:2007:KYE:1362903.1362910} present an attack on credit card payment systems,
    which we described in Section \ref{sec:transparent-bridge} as the Transparent Bridge attack.
This attack relies on the ability to perform out-of-band real-time proxying and relaying of messages between two parties.
Drimer et al. implement this attack against EMV (``Chip and Pin'') credit cards, demonstrating its practicality.
They recommend defending against such attacks via distance bounding,
    essentially measuring round-trip communication timing to detect any delays introduced through the relaying of messages.
Such a defense is reasonable when reading responses directly from chip I/O (as in EMV credit card transactions),
    but does not lend itself well to responses generated by a multitasking computational device such as a smart phone,
    where delays can be variable depending on unrelated software.

In \cite{francis2010practical}, Francis et al. find that out-of-band real-time proxying and relaying of messages is possible over NFC.
To demonstrate this, they demonstrate two NFC devices communicating over a distance much larger than NFC range,
    by using two additional phones (relaying NFC messages over Bluetooth).
While Drimer et al. demonstrated the Transparent Bridge attack with EMV credit cards,
    this result indicates that the attack applies to contactless credit cards as well.
Francis et al. propose to use location information such as GPS coordinates in order to detect and defend against this relaying of messages,
    which in turn would render the Transparent Bridge attack infeasible.
However, location information can be unreliable or unavailable, and as such, one cannot rely on its availability and correctness.
Furthermore, passive NFC tags such as physical contactless credit cards do not have access to location information.

By contrast to \cite{francis2010practical} and \cite{Drimer:2007:KYE:1362903.1362910},
    our approach does not seek to detect or prevent attacks relying on the proxying or relaying of information, choosing instead to render them impotent.

In \cite{eun2013conditional}, Eun et al. explore the issue of privacy in the face of NFC eavesdroppers, considering mobile payments as a case study.
They suggest the creation of an NFC-SEC protocol, complete with key-exchange and public key cryptography, including requirements of
    \emph{unobservability} (an individual transaction may not be distinguishable from other transactions) and
    \emph{unlinkability} (two transactions from the same card may not be identifiable as such), while still maintaining
    \emph{traceability} (it must be possible to ascertain who generated a given set of data in order to troubleshoot problems which may arise).
Eun et al. approach this problem from a clean slate perspective, and do not constrain themselves to making use of existing infrastructure,
    imposing a significant barrier to adoption where existing infrastructure has been deployed.
Furthermore, the \emph{lack} of unlinkability of current credit card transactions is profitable to retailers.
As such, a clean-slate unlinkable protocol is unlikely to see adoption for contactless credit card processing.