Near Field Communication (NFC) is short-range wireless channel which permits two devices to communicate wirelessly,
    allowing for arbitrary computation on both devices.
A distinctive feature of NFC is that one of the two devices may eschew having its own power source (termed a ``passive device''),
    drawing power from the other device in order to perform computation and respond \cite{nfcspec}.
This feature makes NFC particularly useful for use in contactless integrated circuit cards (often termed ``Proximity Cards'') \cite{nfciso},
    where a person may authenticate through demonstrating ownership of a physical token (by bringing it close to a reader).
Such applications are commonly used for access control to buildings, parking garages, and public transit \cite{finvzgar2011use, roberts2006radio, tamrakar2011identity, weinstein2005rfid}.

These same properties also make NFC an attractive channel for use in contactless payment protocols,
    of which common examples include MasterCard's PayPass\textsuperscript{TM} \cite{paypass} and Visa's payWave\textsuperscript{TM} \cite{paywave}.
Use of NFC has also been standardized more generally in EMV's Contactless Specifications for Payment Systems \cite{emv}.
