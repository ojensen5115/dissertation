\section{Security for Credit Card Protocols}
\label{sec:intro-approach-security}

The attacks on the existing contactless credit card protocol rely on a relatively small number of design flaws in the existing contactless credit card protocol.
With these weaknesses in mind, we determine several properties which a secure payment protocol should maintain:

\begin{description}
    \item[Transmitted data should have limited utility.]
        In the existing protocol, any NFC-capable device may claim to be a point of sale, and solicit a credit card for its details.
        It is infeasible for a credit card to authenticate every point of sale,
        Thus, a secure protocol should ensure that credit card responses cannot be replayed to a true point of sale.
    \item[No sensitive information should be transmitted in cleartext.]
        In the existing protocol, any NFC antenna within a small (but non-negligible) range of a legitimate transaction may capture sensitive credit card information.
        With specialized equipment, this area is significantly bigger\cite{brown2013evaluating}.
        Thus, a secure protocol should ensure that no data with a lifetime extending beyond this particular transaction be transmitted.
    \item[Transmitted data should authorize a specific purchase.]
        In the existing protocol, a credit card response can be considered as a key to authorize an arbitrary purchase.
        A retailer may alter the price before issuing the bank, or may even pass the response to another party entirely.
        Thus, a secure protocol should ensure that credit card response data can be used only for the authorized charge amount.
\end{description}

Our first protocol, described in Chapter \ref{cha:external} and called the Externally Secure CC Protocol, takes the first two properties into account,
    drawing much inspiration from challenge-response protocols which achieve similar results.
Two additional considerations render a standard challenge-response unsuitable.
First, limited computational abilities on a credit card render typical response-generating functions
    (such as cryptographic hash functions) computationally infeasible.
Second, while it is the point of sale which issues (and thus generates) the challenge, it is not itself capable of validating credit card data.
We use stepwise refinement to determine necessary properties for a response function, and propose an implementation making use of only minimal computaiton on the card.
By ensuring that transmitted data has limited utility and that no sensitive information is transmitted in cleartext,
    the Externally Secure CC Protocol defends against outsider attacks.

Our second protocol, described in Chapter \ref{cha:secure} and termed the Secure CC Protocol, adds consideration to the third attribute above.
Designed for Electronic Wallet applications such as Android Pay or Apple Wallet, it is backwards-compatible with physical contactless credit cards.
By leveraging the more powerful computational abilities of Electronic Wallet applications, the computational restrictions precluding cryptographic hash functions are relaxed.
Furthermore, the interactive interface allows the cardholder and (virtual) credit card to communicate.
We therefore augment the Externally Secure CC Protocol to confirm the purchase price with the cardholder,
    and bind the price to the card's response using a cryptographic hash function.
By ensuring that transmitted data authorizes only a purchase of a confirmed given price,
    the Secure CC Protocol defends against malicious retailer attacks.

\section{Privacy for Credit Card Protocols}
\label{sec:intro-approach-privacy}

In Chapter \ref{cha:unlinkability}, we discuss the the benefits and drawbacks associated with an additional property: \emph{linkability}.
This property allows a retailer to identify multiple purchases made with the same credit card.
Naturally, any protocol which transmits a credit card number (or any other unique and constant identifier) maintains this property.
This includes all protocols discussed thus-far.

Linkability is valuable to retailers, as it enables the construction of purchasing profiles,
    which can then be used for marketing purposes or sold to interested third parties.
However, this behavior may be undesirable to consumers concerned about their privacy:
    purchasing habits can reveal extremely sensitive information \cite{targetpregnant}.

Our third protocol, described in Chapter \ref{cha:unlinkable_design} and termed the Unlinkable Electronic Wallet Protocol,
    renders retailers unable to identify purchases made using the same credit card (in addition to protecting customers from malicious outsiders and malicious retailers).
While the Externally Secure CC Protocol and the Secure CC Protocol require significant retailer cooperation (in modifying or replacing all point of sale devices),
    the Unlinkable Electronic Wallet protocol makes use of existing point of sale infrastructure,
    requiring no modification to point of sale devices and thus no retailer cooperation.
