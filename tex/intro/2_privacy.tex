\section{Privacy Concerns}
\label{sec:intro-privacy}

In Chapter \ref{cha:unlinkability}, we discuss the benefits and drawbacks associated with an additional property: \emph{linkability}.
This property allows a retailer to identify multiple purchases made with the same credit card.
Naturally, any protocol which transmits a credit card number (or any other unique and constant identifier) maintains this property.
This includes all protocols discussed thus far.

Linkability is valuable to retailers, as it enables the construction of purchasing profiles,
    which can then be used for marketing purposes or sold to interested third parties.
However, this behavior may be undesirable to consumers concerned about their privacy:
    purchasing habits can reveal extremely sensitive information \cite{targetpregnant}.

Our third protocol, described in Chapter \ref{cha:unlinkable_design} and termed the Unlinkable Wallet Protocol,
    renders retailers unable to identify purchases made using the same credit card (in addition to protecting customers from malicious outsiders and malicious retailers).
While the Externally Secure CC Protocol and the Secure CC Protocol require significant retailer cooperation in modifying or replacing all point of sale devices,
    the Unlinkable Electronic Wallet protocol makes use of existing point of sale infrastructure,
    requiring no modification to point of sale devices and thus no retailer cooperation.
