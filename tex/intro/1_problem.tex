\section{Security Issues}
\label{sec:intro-problem}

The protocol in use by contactless credit cards falls short of addressing even basic security issues.
It employs no authentication, allowing any device to effectively pose as a Point of Sale.
Furthermore, it transmits sensitive information (the credit card number, expiration date, etc.) wirelessly in cleartext.
This transmission of data is then considered sufficient to allow an arbitrary charge to occur.
This leads to a number of potential security vulnerabilities.

For example, skimming a credit card is easy.
Simply instruct an NFC-capable phone to send out solicitation messages claiming to be a point of sale, and any contactless credit cards within range will respond with their information.
Indeed, there is an Android application \cite{NFCProxy} which simplifies this process down to clicking a button.
This skimmed data can then be replayed to a real point of sale in order to perform a purchase on behalf of the skimmed credit card.

Eavesdropping is likewise easy.
NFC range is quite short, greatly limiting the possible location of an eavesdropper.
However, constructing an eavesdropping antenna out of an inexpensive NFC tag is a simple procedure, as described in Section \ref{sec:insecure-attacks}.
Concealing such an antenna within a few centimeters (or even attached to) of a Point of Sale is sufficient to harvest credit card information,
    due to the plaintext nature of the messages being sent.

A credit card's response to a solicitation is likewise over-broad in how it may be used:
    while a Point of Sale may display a given price to the customer, there is no protocol-level assurance that the card will be charged the price on the screen.
Instead, a credit card's response provides a point of sale with the ability to charge any arbitrary single purchase to the credit card.
This charge need not even come from the same retailer or establishment.
