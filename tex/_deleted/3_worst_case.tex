\subsection{Worst Case Analysis}
\label{sec:analysis_worstcase}

A point of sale may do rudimentary checking on a Card Information message before issuing a Charge Request message.
In practice, 28 numeric digits does not imply $10^{28}$ possible messages, due to several constraints which may be verified without involving the bank.

\subsubsection*{Constraint: Luhn Algorithm Checksum}

Credit card numbers include a checksum formula known as the Luhn algorithm, commonly used to check for typographical errors when entering credit card details.
This constrains the final digit of a 16-digit credit card.
As such, a point of sale may choose to reject a Card Information message whose last digit of its credit card number does match.
This reduces the number of admissible Card Information messages by a factor of 10.

\subsubsection*{Constraint: Valid Expiration Dates}

A credit card's expiration date is not valid if it is in the past.
Furthermore, credit cards are not typically issued with an expiration date further than four years into the future.
As such, a point of sale may choose to reject a Card Information message whose 4-digit expiration date is not one of 48 possible month/year combinations.
This reduces the number of admissible Card Information messages by a factor of $\frac{10^4}{48} \approx 208$.

\subsubsection*{Constraint: BIN}

The first 4-6 digits of a credit card are typically reserved as a numeric bank identifier number, commonly referred to as a BIN.
This information may be considered redundant to a certain degree, since NFC Card Information messages contain the bank name.
Nonetheless, the point of sale may impose constraints on these six digits.
In the worst case, let us assume that there is only one permissible BIN usable by the Unlinkable CC Protocol.
This reduces the number of admissible Card Information messages by a factor of $10^6$.

\subsubsection*{Remaining Message Space}

As a result of the above constraints, worst-case assumptions leave us with a total of $48 * 10^{17}$ possible Card Information messages.
Each Card Information message may then encode just 62 bits, as compared to 98 bits when Card Information messages are unconstrained.
By shrinking the Price Hash component to 10 bits (representing a one-in-1024 chance for a malicious retailer to alter a price),
  we are left with a Price Hash of 50 bits.

Repeating the same analysis with a 50-bit price hash, a Wallet Server may support up to $2^49$ (approximately 560 trillion) credit cards
  while keeping card registration computationally easy as defined in Section \ref{sec:analysis}.
Once more assuming 10 million registered cards, the expected number of transactions before a collision occurs is $\frac{2^{50}}{10^8}$,
  or approximately 112 million transactions.

This figure is no longer negligible, highlighting the importance of supporting a collision mitigation strategy.
However, the impact remains small:
  a collision affect only those two cards with colliding account hash values,
  and results in both cards becoming inoperative until at least one of the requisite Electronic Wallet applications regains access to the Internet.
