\section{Anatomy of a Credit Card}

A credit card is a device composed of multiple interfaces, any of which may be used to authorize payments on behalf of the cardholder.
The common interfaces seen on credit cards today are as follows:

\begin{enumerate}
\item \textbf{Embossed}

The \emph{embossed} interface consists of all data which is embossed on the credit card.
It was primarily used to transfer credit card information to carbon paper at retailers, before magnetic swiping became prevalent.
While mostly obsolete, it is still used in some markets, when more conventional credit card processing is not convenient or available.
For example, this interface is commonly used for taxi payments in countries where wireless infrastructure is scarce.
The data transmitted through this interface consists of the credit card number, expiration date, and cardholder name.

\item \textbf{Magstripe}

The \emph{magstripe} interface consists of all data which can be read from the magnetic strip on the back of a credit card.
The magstripe interface is commonly used at retailers, gas stations, and ATMs, when the card is swiped.
The data transmitted through this interface consists primarily of the credit card number, expiration date, cardholder name, and \emph{CVV1}.
The CVV1, known as the \emph{Card Verification Value 1}, is simply a number encoded on the magnetic strip that is not otherwise present on the card.
It prevents a would-be fraudster from duplicating the magnetic strip of a credit card, provided that they are unable to read the magnetic stripe of the original.

\item \textbf{Visual}

The \emph{visual} interface consists of all data which can be read visually from the credit card.
The visual interface is commonly used on the Internet or over the phone, when a customer is asked for credit card details.
The data transmitted through this interface consists primarily of the credit card number, expiration date, cardholder name, and \emph{CVV2}.
The CVV2, known as the \emph{Card Verification Value 2}, is the three-digit number on the back of the card.
It combats fraud by requiring information from both sides of the credit card, thwarting would-be fraudsters armed with a photograph of the card.
The CVV2 is also subject to regulation, in that no party is permitted to record it under any circumstances.
As a result, its inclusion is also commonly used as ``proof of presence'' for the physical card itself.

\item \textbf{NFC}

The \emph{NFC} interface consists of the data and computation involved when reading the credit card via NFC reader.
The NFC interface is commonly used in ``contactless'' transactions, when a point of sale supports tapping the credit card to a reader rather than swiping it.
It is also emulated by mobile payment applications such as Google Wallet and Apple Pay, allowing an (authorized) smart phone to pose as a credit card.
The data transmitted through this interface primarily consists of the credit card number, expiration date, issuing bank's name, and \emph{iCVV}.
The iCVV, known as the \emph{integrated Card Verification Value}, is a dynamically generated element in a pseudorandom sequence.
This sequence must be presented in order, is predictable only to the bank, and any individual value valid only for a single purchase.
The iCVV combats fraud by standing in the way of anyone possessing an NFC reader (e.g. a smart phone) from duplicating an NFC credit card by simply being near enough to read it.

\item \textbf{Chip}

The \emph{chip} interface consists of the data and computation involved when inserting the card into a chip reader.
The chip interface is commonly used at retailers and banks.
It is a newer interface, and despite seeing widespread use in the European Union, it has only recently seen adoption in the United States.
Businesses in the United States are now incentivized to support use of the chip interface (and when possible to require it over the magstripe interface), through differing fraud liability structures.
\end{enumerate}
