The issue involved is described through a series of strawman solutions.

\begin{description}
\item{\textbf{Symetric Key Cryptography:}}
The customer encrypts all identifying data with a secret key, known only to that customer and the bank.
We assume that the cryptography scheme used incorporates ``semantic security'', meaning that multiple encryptions of the same data yields
  different ciphertexts.
In so doing, the bank may still identify the customer, but the retailer can not.
However, in order to decrypt the datagram, the bank must know which customer's key to use for decryption.
As a result, includes a key identifier which the bank can use to retrieve a customer's key.
This scheme is not unlinkable: the retailer may build profiles based on key identifiers.
\item{\textbf{Symetric Key Cryptography with a Global Key:}}
All customers encrypt all identifying information with a common key.
In so doing, the bank may easily decrypt the datagrams, because it knows which key to use.
However, in order for any customer to be able to encrypt its information, the key must be readily available.
As a result, the scheme is not unlinkable: the retailer may simply acquire the key for itself, and decrypt the identifying information.
\item{\textbf{Public Key Cryptography:}}
This problem seems like an ideal use of public key cryptography:
we would like anybody to be able to craft confidential messages to a single party, without being able to decrypt each others' messages.
However, retailers would prever to prevent transactions from becoming unlinkable, and ultimately choose which messages to accept.
Public key cryptography requires significantly larger messages to be sent, and thus could not easily be corraled into existing message formats.
As a result, retailers can prevent unlinkable purchases simply by choosing not to support this new scheme.
\end{description}





It is with this in mind that we define a new property: \emph{unlinkability}.
Informally, can be thought of as making it impossible for such a profile to be built.
More formally, we define it in terms of an adversarial game:

\begin{enumerate}
\item There exist two credit cards, $c_0$ and $c_1$
\item The retailer has access to all previous purchase data, including an arbitrary number of past purchases from both $c_0$ and $c_1$
\item The customer flips a fair coin to generate $b \in \left\{0 | 1\right\}$, and performs a purchase $p$ using card $c_b$
\item The retailer examines the purchase $p$ and outputs $b'$, its guess at the value $b$
\item Purchase $p$ is said to be \emph{unlinkable} if the probability that $b = b'$ does not exceed one half
\end{enumerate}