In this chapter, we design the Unlinkable CC Protocol, achieving our three primary goals:
supporting unlinkabile payments,
allowing the use of any credit card,
while using existing point of sale infrastructure.

We begin by making several strong assumptions.
In particular, we assume that:
\begin{itemize}
\item The retailer is not malicious
\item The Wallet Application is connected to the Internet at all times
\item The point of sale experiences no communication failures
\end{itemize}
As we refine the protocol, we will erode these assumptions away to yield a secure and resilient protocol.

As foreshadowed in the previous chapter, this protocol will span multiple credit card interfaces.
In particular, our protocol design will make use of both the \emph{NFC}\footnote{
	When using the NFC interface, credit cards are authenticated via card number, expiration date, and iCVV.
} and the \emph{visual}\footnote{
	When using the visual interface, credit cards are authenticated via cardholder name, card number, and expiration date,
	optionally accompanied by the CVV2 (the 3-digit number on the back of the card) and/or billing address.
} interfaces.
As a result, message names in this chapter will identify the relevant interface where appropriate.

The Unlinkable CC Protocol is designed solely for use with Electronic Wallets, and cannot be used with physical credit cards.
Thus, the first step for a customer is to download the Wallet Application and register a credit card.

The card registration process is simple.
The customer launches the Wallet Application enters the credit card details for the card he wishes to register.
These details consist of the cardholder name, card number, expiration date, and billing address.
The Wallet Application then sends this data in a Registration message to a new principal: the Wallet Server.
This Registration message is transmitted securely over the Internet.

The customer may then use any card registered through the Wallet Application to make unlinkable transactions.
