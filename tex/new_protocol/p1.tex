\section{Basic Protocol}
\label{unlinkable-design-1}

When the Wallet Server receives a Registration message, it stores the card details and generates a random 93 bit \emph{token}.
This token is sent to the Wallet Application, for use in its next transaction.

When the customer wishes to make a purchase, the Wallet Application selects a registered card and begins listening for Solicitation messages.
The protocol then operates as follows:
\begin{enumerate}
\item The point of sale sends a Solicitation message to the Wallet Application over the NFC channel.
\item The Wallet Application looks up the \emph{token} which was issued for the card selected by the customer.
    It then converts the stored 93-bit token into a 28-digit number \emph{k}
    (note that 28 digits is sufficient to store a 93 bit value: $log_{10}(2^{93}) \approx 27.995$).
    Finally, it responds with an \emph{NFC} Card Information message.
    This message has the following fields:
    \begin{itemize}
    \item \textbf{Card number:} the first 16 digits of \emph{k}
    \item \textbf{Expiration date:} the subsequent 4 digits of \emph{k}
    \item \textbf{iCVV:} the remaining 8 digits of \emph{k}
    \item \textbf{Bank name:} the Wallet Server
   	\end{itemize}
\item The point of sale constructs an \emph{NFC} Charge Request message from the (virtual) Card number, Expiration date, iCVV, and the price it wishes to charge.
	This message is sent to the bank named in the Card Information message.
    As a result, the Charge Request essage is directed to the Wallet Server and \emph{not} a bank.
    Note that from the perspective of the point of sale, the Wallet Server appears to be a bank like any other.
\item The Wallet Server reconstructs \emph{k} from the Carge Request message, and computes the 93-bit token it represents.
    The Wallet Server then searches its database for this token, to identify the card used in this transaction.
    If no result is found, the Wallet Server sends a ``declined'' Acceptance message to the point of sale, and aborts the protocol.
	Otherwise, the stored card details are retrieved from the Wallet Server's database.
    The Wallet Server then sends a \emph{visual} Charge Request to the card's bank, with the following fields:
    \begin{itemize}
    \item Cardholder name
    \item Card number
    \item Expiration date
    \item Billing address
    \end{itemize}
    Note that unlike the information sent in the Card Information message, this data reflects the actual credit card information.
\item The bank receives the \emph{visual} Charge Request from the Wallet Server.
    The bank processes this transaction in the same was as it currently does in the Insecure CC Protocol described in Chapter \ref{cha:insecure}.
    Finally, it responds to the Wallet Server with an Acceptance message indicating whether the charge has been accepted.
\item The Wallet Server forwards the bank's Acceptance message to the point of sale.
\end{enumerate}

Note that in this protocol, the Wallet Server has a dual role:
to the point of sale, the Wallet Server appears to be a bank, while to the bank, the Wallet Server appears to be a point of sale.

This protocol is unlinkable,
    because the information received by the point of sale consists solely of a random value, accompanied by a bank name identifying the Wallet Server.
Any credit card can be used in this protocol,
    because the bank receives \emph{visual} Charge Request messages, and all credit cards support the visual interface.
Finally, this protocol uses existing infrastructure,
    in that the behaviors of the point of sale and the bank are unchanged from those of the Insecure CC Protocol.
