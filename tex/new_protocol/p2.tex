\section{Augmented Unlinkable CC Protocol: Protecting against Malicious Retailers}
\label{sec:unlinkable-design-2}

This protocol is very similar to the Basic Unlinkable Protocol described in Section \ref{sec:unlinkable-design-1}.
In the Basic protocol, a 93-bit token was used to identify a customer to the Wallet Server.
In the Augmented protocol, the token consists of only 80 bits, leaving 13 bits left over to bind the price to the token.

When the Wallet Server receives a Registration message, it stores the card information and associates a card identifier \emph{ID} with this record.
It also generates a secret key \emph{K} associated with this card.
In the Augmented protocol, the Wallet Server responds to the Registration message with this identifier \emph{ID} and the key \emph{D}.
The Wallet Application stores these values securely, associating them with the credit card being registered.
The protocol then operates as follows, and is illustrated in Figure \ref{fig:unlinkable-1}:

\begin{enumerate}
    \item The point of sale displays the price to pay on its screen.
    \item The customer selects a credit card in the Wallet Application, and keys in the price to pay.
    \item The point of sale sends a Solicitation message to the Wallet Application over the NFC channel.
    \item The Wallet Application looks up the 80-bit token which was issued for the card selected by the customer.
        It then calculates a 13-bit price hash \emph{h\textsubscript{p} = H(token, price)}$|$\textsuperscript{13}.
        The Wallet Application then concatenates the token and price hash into a 93-bit value,
        and converts this value into an \emph{NFC} Card Information addressed to the point of sale using the same procedure as in the Basic Protocol.
    \item The Point of Sale receives the Card Information message, and constructs an \emph{NFC} Charge Request from it and the price it wishes to charge.
        This Charge Request message is sent to the Wallet Server as before.
    \item The Wallet Server reconstructs the 93-bit value from the Charge Request message, and splits it into the 80-bit token and 13-bit price hash.
\end{enumerate}